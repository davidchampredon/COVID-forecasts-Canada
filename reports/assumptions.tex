\subsection*{Assumptions and Caveats}
\begin{itemize}

\item \textbf{Simplification.} This forecast is based on a mathematical model of SARS-CoV-2 transmission. Although the model attempts to capture the overall dynamic of the epidemic, it is a simplification of the reality.

\item \textbf{Data Quality.} To perform forecast, the model must be ``calibrated'' to observations. The quality of the forecasts is directly impacted by the quality of the observed data. 
In particular, the number of COVID-19 positive tests publicly available on Federal or Provincial websites is used as a proxy for the true, unobserved, incidence of infections. Thus, the model is impacted (among many other issues) by the saturation of tests performed, bias from the demographics tested (e.g., travellers, symptomatic and hospitalized persons), reporting delays and errors.

\item \textbf{Technical Details.} The mathematical model of transmission is based on a stochastic renewal equation that incorporates uncertainty for the transmission process as well as for the observation process. The documentation and implementation code will be available online soon.


\end{itemize}

