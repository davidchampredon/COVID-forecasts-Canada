\documentclass[10pt]{article} 
\usepackage[letterpaper, 
			top=20mm, 
			bottom = 25mm,
			left = 25mm,
			right = 25mm]{geometry}
\usepackage[parfill]{parskip}
\usepackage{graphicx}
\usepackage{amssymb}
\usepackage{xcolor}
\usepackage{xspace}
\usepackage[useregional]{datetime2}
\usepackage{lastpage}

\usepackage{array}
\newcolumntype{C}[1]{>{\centering\let\newline\\\arraybackslash\hspace{0pt}}m{#1}}


\usepackage[]{hyperref}
\hypersetup{
    colorlinks,
    linkcolor={black},
    citecolor={blue!50!black},
    urlcolor={blue!99!black}
}

\usepackage[font={small,it}]{caption} 
\captionsetup[table]{skip=10pt}

\def\asof{\today}
% \def\asof{2020-04-09} % <--- Manual override 

% ===== HEADERS / FOOTERS =====
\usepackage{fancyhdr}
\setlength{\headheight}{15.2pt}
\lhead[]{\textsf{COVID19 Forecasting Report}}
\rhead[]{\asof}
\renewcommand{\headrulewidth}{0.8pt}
\fancyfoot[L]{David Champredon}
\fancyfoot[C]{\scriptsize{\DTMnow}}
\fancyfoot[R]{\thepage / \pageref{LastPage}}
\renewcommand{\footrulewidth}{0.4pt}
\pagestyle{fancy}
% ==============================


% ====== MACROS ========

%\newcommand{\prov}{ON} % <-- for DEBUG

\newcommand{\pathOut}{../src/out/}
\newcommand{\scen}[1]{\textbf{\textsf{#1}}}
\newcommand{\Ro}{\ensuremath{\mathcal{R}_0}\xspace}
\newcommand{\Rt}{\ensuremath{\mathcal{R}_t}\xspace}
\newcommand{\warning}[1]{\textbf{\textcolor{red}{#1}}\xspace}

% ==============



\begin{document}  
\thispagestyle{empty}

\Huge{\textsf{COVID19 Forecasting Report}}\\

%\begin{table}[ht]
\normalsize
\begin{tabular}{ll}
\textsf{As of date} & \textsf{: \asof} \\
\textsf{Region} & \textsf{: \prov} \\
\textsf{Contact} & \textsf{: David Champredon (\texttt{david.champredon@gmail}) - Western University, London, ON} \\
\end{tabular}
%\end{table}

\vspace*{15mm}

% =================================
% ========== FOREWORDS ============
% =================================

\normalsize

\subsection*{Assumptions and Caveats}
\begin{itemize}

\item \textbf{Simplification.} This forecast is based on a mathematical model of SARS-CoV-2 transmission. Although the model attempts to capture the overall dynamic of the epidemic, it is a simplification of the reality.

\item \textbf{Data Quality.} To perform forecast, the model must be ``calibrated'' to observations. The quality of the forecasts is directly impacted by the quality of the observed data. 
In particular, the number of COVID-19 positive tests publicly available on Federal or Provincial websites is used as a proxy for the true, unobserved, incidence of infections. Thus, the model is impacted (among many other issues) by the saturation of tests performed, bias from the demographics tested (e.g., travellers, symptomatic and hospitalized persons), reporting delays and errors.

\item \textbf{Technical Details.} The mathematical model of transmission is based on a stochastic renewal equation that incorporates uncertainty for the transmission process as well as for the observation process. The documentation and implementation code is available at:

\texttt{\href{https://github.com/davidchampredon/COVID-forecasts-Canada}{github.com/davidchampredon/COVID-forecasts-Canada}}.


\end{itemize}





\newpage


% ======================================
% ========== PAST & CURRENT ============
% ======================================


\section*{Past and Current Dynamic}




The proxy for incidence of infections used in this document are reported tests compiled and curated by Michael Li at \href{https://github.com/wzmli/COVID19-Canada}{github.com/wzmli/COVID19-Canada}.




\subsection*{Doubling Time}

The \emph{daily} confirmed cases are used to estimated the doubling time of the epidemic (\autoref{fig:doubling}). 
The doubling time is the expected number of days it takes for the number of daily confirmed cases to double. The higher the doubling time, the slower the epidemic progression is.
The evolution of the doubling time quantifies any acceleration or deceleration of the outbreak. In particular, it can reveal the impact (or the lack of) of an intervention. 

\begin{figure}[h!]
\begin{center}
\includegraphics[width = 1.1\textwidth]{../src/plots/plot-data-\prov.pdf}
\caption{Left Panel: Reported tests and deaths. The grey segment represents, for each time series, the portion of the data used to calculate the doubling time. Right Panel: Evolution of the doubling time when calculated from a sliding observation window starting 10 days before the calculation date. The shaded area represents the 95\% confidence interval. The vertical scale is cropped at 60 days for clarity.}
\label{fig:doubling}
\end{center}
\end{figure}


%\begin{figure}[h!]
%\begin{center}
%
%\includegraphics[width = 0.45\textwidth]{../src/plots/plot-data-growth-cone-\prov.pdf}
%\hspace{5mm}
%\includegraphics[width = 0.45\textwidth]{../src/plots/plot-data-dbl-slide-\prov.pdf}
%
%\caption{
%Left Panel: New positive tests reported daily. The dashed line shows, for reference, the trajectories of cases doubling every 2 and 4 days.
%Right Panel: Doubling time (in days). The doubling times were estimated with a linear regression on the count of the log number of daily confirmed cases. }
%\label{}
%\end{center}
%\end{figure}


\subsection*{Estimated effect of social distancing}


The effect of social distancing is estimated by fitting an epidemic model that has a discontinuity in the transmission rate on March 20 (see \autoref{tab:scenarios} for the definition of scenario \scen{ISO1} later in this report). The size of the discontinuity is calibrated on positive tests observed after March 20 (declaration of the state of emergency in Ontario was on March 17). 

\renewcommand{\arraystretch}{1.5}
\begin{table}[h!]
\begin{center}
\begin{tabular}{lc c}
\hline
\bf  & \bf Mean & \bf 95\%CI \\
\hline


Transmission rate reduction & 
\input{../src/out/lambda-mean-\prov-ISO1.txt}\%&
( \input{../src/out/lambda-lo-\prov-ISO1.txt}\% --
\input{../src/out/lambda-hi-\prov-ISO1.txt}\%)\\


\hline
\end{tabular}
\end{center}
\end{table}



\newpage

\subsection*{Effective Reproduction Numbers}

The \emph{effective} reproduction number  -- traditionally noted \Rt \, -- takes into account the depletion of susceptible individuals in the population: it is defined as the average number of secondary transmissions from an infectious individual (not necessarily in a fully susceptible population). The subscript $t$ represents the time when \Rt is calculated.
When the value of \Rt decreases below 1, this signals the epidemic is decelerating.
The evolution of \Rt since the date when enough data is available for each region is shown in \autoref{fig:Rt}. 



\begin{figure}[h!]
\begin{center}
\includegraphics[width = 1\textwidth]{../src/plots/plot-data-Rt.pdf}
\caption{Evolution of the effective reproduction number \Rt (blue line). The green area marks the threshold level of $\Rt \leq 1$ where the epidemic starts to decelerate. The labelled number indicates the current mean estimate of \Rt.}
\label{fig:Rt}
\end{center}
\end{figure}




% ======================================
% ============== FORECASTS =============
% ======================================

\newpage


\section*{Forecasts}



Once the mathematical model is fitted to observed data, the future trajectory of incidence is simulated. Uncertainty about the values of model parameters, as well as the transmission and reporting processes, are taken into account and propagated in the forecasts. 
Control measures are simulated in 3 different scenarios.



\subsection*{Intervention Scenario Definitions}



\renewcommand{\arraystretch}{1.7}

\begin{table}[h!]
\caption{Intervention scenarios used in the forecasts.}
\begin{tabular}{l p{0.8\textwidth}} %

\hline
\bf Name & \bf Description \\
\hline
\scen{Baseline} & This is the baseline scenario, assuming the transmission rate will remain the same as it was before the first social distancing intervention. This scenario is fitted to positive tests reports until March 20th.\\

\scen{ISO1} & This is the scenario we are currently in, where strong social distancing measures took effect after March 20th and are kept as is for 6 months.\\

\scen{ISO2} & This is the scenario we are currently in, where strong social distancing measures took effect after March 20th. Social distancing will gradually be relaxed from May 15 until July 15. At that point, the transmission rate will return to the same value as before the social distancing measures. \\
\hline
\end{tabular}
\label{tab:scenarios}
\end{table}



\vspace*{5mm}
Projections are presented for three type of outcomes:

\begin{tabular}{ll}
\textsf{hosp} & : admissions to hospital \\
\textsf{critical} &: transfer to critical care of hospitalized cases (incidence)\\
\textsf{death} &: deceased cases
\end{tabular}



\vspace*{5mm}

\subsection*{Model Parameters}

\renewcommand{\arraystretch}{1.5}
\begin{table}[h!]
\begin{center}
\begin{tabular}{l c c}
\hline
\bf Parameter & \bf Value & \bf Source \\
\hline
initial prop. susceptible  & 80\% & assumption \\
prop. confirmed cases out of true incidence & 25\% (95\%CI: 15-35\%) & \href{https://cmmid.github.io/topics/covid19/severity/global_cfr_estimates.html}{CMMID} \\
prop. confirmed to hospitalized & 12.5\% & PHO daily reports \\
prop. hospitalized to critical care & 39\% & PHO daily reports \\
prop. critical to death & 65 -- 85\% & fitted on death reports \\
generation interval  (mean, sd) & $m=4.5$ days ; $\sigma=2.6$ & various studies \\
contact heterogeneity & high ($2\leq\alpha\leq 6$) & assumption\\
\hline
\end{tabular}
\end{center}
\end{table}%




\newpage

\subsection*{Total Burden Projections}



The total burden in \autoref{fig:fcstCum} considers the cumulative number of patients that have been admitted to hospital, admitted in critical care, and who died throughout the full duration of the simulated epidemics.

\begin{figure}[h!]
\begin{center}
\includegraphics[width = 0.99\textwidth]{../src/plots/plot-analyze-cum-\prov.pdf}
\caption{Estimated total burden over the course of simulated epidemics. \textsf{hosp}: hospitalized cases; \textsf{critical}: hospitalized cases in critical care; \textsf{death}: number of deaths. The lower panel is an example of a  probabilistic interpretation of the upper panel.}
\label{fig:fcstCum}
\end{center}
\end{figure}


\newpage 

\subsection*{Peak Timing Projections}



The simulated epidemics allow us to forecast the time when the daily number of hospital admissions peaks. \autoref{fig:fcstPkt} shows the possible peak date ranges for each scenario.

\begin{figure}[h!]
\begin{center}
\includegraphics[width = 0.99\textwidth]{../src/plots/plot-analyze-pkt-\prov.pdf}
\caption{Forecast for the peak time of hospital admissions.}
\label{fig:fcstPkt}
\end{center}
\end{figure}

\newpage

\subsection*{Peak Daily Intensities Projections}



Similarly as the peak time forecasts, the peak intensity for the three outcomes (hospitalized, critical care and death) can be projected. \autoref{fig:fcstPkv} shows the possible peak intensity ranges for each scenario and outcome.

\begin{figure}[h!]
\begin{center}
\includegraphics[width = 0.99\textwidth]{../src/plots/plot-analyze-pkv-\prov.pdf}
\caption{Daily peak projections. \textsf{hosp}: hospitalized cases; \textsf{critical}: hospitalized cases in critical care; \textsf{death}: number of deaths.}
\label{fig:fcstPkv}
\end{center}
\end{figure}




\newpage

\subsection*{Short-Term Projections}

The probability that the sum of the confirmed cases over the \emph{next} 7 days will be higher than the sum of the \emph{past} 7 days is shown in \autoref{fig:probaHigher} for  scenario \scen{BASELINE} and \scen{ISO1} (scenarios \scen{ISO1} and \scen{ISO2} are equivalent before May 15).

\begin{figure}[h!]
\begin{center}
\includegraphics[width = 0.8\textwidth]{../src/plots/plot-analyze-probaHigher-\prov.pdf}
\caption{For each scenario, probability that the sum of confirmed cases for next week (7 days) will be higher than last week. }
\label{fig:probaHigher}
\end{center}
\end{figure}


The mean cumulative number of deaths, as well as the probability that this number will be above 1,000 by May 15, is displayed in the Table below for %\scen{BASELINE} and
\scen{ISO1} (scenarios \scen{ISO1} and \scen{ISO2} are equivalent before May 15). The probability distribution is shown in \autoref{fig:cumdeathShort}.


\renewcommand{\arraystretch}{1.2}
\begin{table}[h!]
\begin{center}
\begin{tabular}{c r C{44mm}}
\hline
\bf Scenario & \bf Mean & \bf Proba. Cum. Death above 1,000 by May 15  \\
\hline

%\sf BASELINE & 
%\input{../src/out/cumdeath-shortterm-mean-\prov-BASELINE.txt}&
%\input{../src/out/cumdeath-shortterm-proba-above-\prov-BASELINE.txt}\% \\

\sf ISO1 & 
\input{../src/out/cumdeath-shortterm-mean-\prov-ISO1.txt}&
\input{../src/out/cumdeath-shortterm-proba-above-\prov-ISO1.txt}\% \\

%\sf ISO2 & 
%\input{../src/out/cumdeath-shortterm-mean-\prov-ISO2.txt}&
%\input{../src/out/cumdeath-shortterm-proba-above-\prov-ISO2.txt}\% \\

\hline
\end{tabular}
\label{tab:cumdeathShort}
\end{center}
\end{table}




\begin{figure}[h!]
\begin{center}
\includegraphics[width = 0.80\textwidth]{../src/plots/plot-cumdeath-shortterm-\prov.pdf}
\caption{Distribution of the simulated cumulative number of deaths by May 15 according to scenario \scen{ISO1}.}
\label{fig:cumdeathShort}
\end{center}
\end{figure}








\end{document}







